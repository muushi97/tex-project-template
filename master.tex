\documentclass[uplatex, 11pt, a4j, dvipdfmx]{jsarticle}

%------------------------------------------------------------------------
% プリアンブル読み込み
%------------------------------------------------------------------------
%------------------------------------------------------------------------
% usepackages
%------------------------------------------------------------------------
% 使用可能な色を増やす
%\usepackage[dvipsnames,svgnames,x11names,dvipdfmx,table,xcdraw]{xcolor}
% 画像読込みを可能に
%\usepackage[dvipdfmx]{graphicx}
\usepackage[dvipdfmx]{}
% レイアウト変更用
\usepackage{geometry}
% 出力用紙サイズのみの変更
\usepackage{bxpapersize}
% セクションのサイズ等を変更 (paragraph があるのでbeamerでは使えない)
%\usepackage{titlesec}
% テーブルのセル内で改行を可能に
\usepackage{tabularx}
% 画像の位置を書いた位置に矯正
\usepackage{here}
% url の表示を可能に
\usepackage{url}
% ソースコード表示用
\usepackage{listings, jlisting}
% otf フォントを使用可能に
\usepackage[uplatex]{otf}
% フォントエンコーディングをT1に変更し、欧文フォントのデフォルトを Latin Modern に変える
\usepackage[T1]{fontenc}
\usepackage{lmodern}
% 図を複数個同一環境にならべたときにキャプションを個別につけられるように
\makeatletter
\let\MYcaption\@makecaption
\makeatother
\usepackage{subcaption}
\captionsetup{compatibility=false}
\makeatletter
\let\@makecaption\MYcaption
\makeatother
% 数式系
\usepackage{amsmath}
\usepackage{diffcoeff}
\usepackage{newtxmath, newtxtext}
%\usepackage{bm}
\usepackage{ascmac}
% hyperlink
%\usepackage[dvipdfmx, bookmarkstype=toc, colorlinks=true, linkcolor=cyan, pdfborder={0 0 0}, bookmarks=true, bookmarksnumbered=true, hypertexnames=false]{hyperref}
%\usepackage{pxjahyper}
% 参照を便利に
\usepackage{cleveref}
% 参照した数式のみに数式番号をつける
\usepackage{autonum}
% 数学系いろいろ
\usepackage{mathtools}
% SI単位用
\usepackage{siunitx}                    % SI

%------------------------------------------------------------------------
% setting for listing
%------------------------------------------------------------------------
\lstset{%
    language={C},
    basicstyle={\small},%
    identifierstyle={\small},%
    commentstyle={\small\itshape},%
    keywordstyle={\small\bfseries},%
    ndkeywordstyle={\small},%
    stringstyle={\small\ttfamily},
    frame={tb},
    breaklines=true,
    columns=[l]{fullflexible},%
    numbers=left,%
    xrightmargin=0zw,%
    xleftmargin=3zw,%
    numberstyle={\scriptsize},%
    stepnumber=1,
    numbersep=1zw,%
    lineskip=-0.5ex%
}
\lstset{%
    language={tex},
    basicstyle={\small},%
    identifierstyle={\small},%
    commentstyle={\small\itshape},%
    keywordstyle={\small\bfseries},%
    ndkeywordstyle={\small},%
    stringstyle={\small\ttfamily},
    frame={tb},
    breaklines=true,
    columns=[l]{fullflexible},%
    numbers=left,%
    xrightmargin=0zw,%
    xleftmargin=3zw,%
    numberstyle={\scriptsize},%
    stepnumber=1,
    numbersep=1zw,%
    lineskip=-0.5ex%
}


%------------------------------------------------------------------------
% change the numbering \equation with (section.equation)
%------------------------------------------------------------------------
\makeatletter
    \renewcommand{\theequation} {%
        %\arabic{chapter}.\arabic{section}.\arabic{equation}}
        \arabic{section}.\arabic{equation}}
    \@addtoreset{equation}{section}
\makeatother



\crefname{figure}{図}{図}
\crefname{subfigure}{図}{図}
\crefname{table}{表}{表}
\crefname{subtable}{表}{表}
\crefname{equation}{式}{式}

\newtheorem{dfn}{定義}
\crefname{dfn}{定義}{定義}

\newtheorem{thm}{定理}
\crefname{thm}{定理}{定理}

\newtheorem{lem}{補題}
\crefname{lem}{補題}{補題}

\newtheorem{coro}{系}
\crefname{coro}{系}{系}

\newtheorem{prop}{命題}
\crefname{prop}{命題}{命題}

\newtheorem{proof}{証明}
\crefname{proof}{証明}{証明}

\newtheorem{ex}{例}
\crefname{ex}{例}{例}

\newtheorem{rem}{注意}
\crefname{rem}{注意}{注意}

\crefformat{page}{#2#1#3{ページ}}%
\crefrangeformat{page}{#3#1#4{〜}#5#2#6{ページ}}%
\crefmultiformat{page}{#2#1#3{ページ}}%
        {, #2#1#3{ページ}}{, #2#1#3}{, #2#1#3{ページ}}%
\crefformat{section}{#2#1#3{節}}%
\crefrangeformat{section}{#3#1#4{〜}#5#2#6{節}}%
\crefmultiformat{section}{#2#1#3{節}}%
        {, #2#1#3{節}}{, #2#1#3}{, #2#1#3{節}}%
\crefformat{subsection}{#2#1#3{節}}%
\crefrangeformat{subsection}{#3#1#4{〜}#5#2#6{節}}%
\crefmultiformat{subsection}{#2#1#3{節}}%
        {, #2#1#3{節}}{, #2#1#3}{, #2#1#3{節}}%
\crefformat{subsubsection}{#2#1#3{節}}%
\crefrangeformat{subsubsection}{#3#1#4{〜}#5#2#6{節}}%
\crefmultiformat{subsubsection}{#2#1#3{節}}%
        {, #2#1#3{節}}{, #2#1#3}{, #2#1#3{節}}%

\crefformat{part}{{第}#2#1#3{部}}%
\crefformat{chapter}{{第}#2#1#3{章}}%

% fig
\newcommand{\fig}[5]{\begin{figure}[#4]\begin{center}\includegraphics[#5]{#1}\caption{#2}\label{fig:#3}\end{center}\end{figure}}

% tab
\newenvironment{tab}[4]{\begin{table}[#1]\centering \caption{#2}\label{tab:#4}\begin{tabular}{#3}}{\end{tabular}\end{table}}

% set
\def\Set#1{\Setdef#1\Setdef}
\def\Setdef#1|#2\Setdef{\left\{#1\,\;\mathstrut\vrule\,\;#2\right\}}%

% sets
\newcommand{\N}{\mathbb{N}}
\newcommand{\Z}{\mathbb{Z}}
\newcommand{\R}{\mathbb{R}}
\newcommand{\Q}{\mathbb{Q}}
\newcommand{\C}{\mathbb{C}}



%------------------------------------------------------------------------
% change font size of section and subsection title
%------------------------------------------------------------------------
%\titleformat*{\section}{\normalsize\bfseries}
%\titleformat*{\subsection}{\normalsize\bfseries}

%------------------------------------------------------------------------
% other setting
%------------------------------------------------------------------------
%\setlength{\columnsep}{3zw}
\setcounter{tocdepth}{3}                       % chenge the depth displayed in the table of contents




\title{自分用texチートシート}
\author{muushi97}
\date{}

% -----------------------------------------------------------------------
% 本文
% -----------------------------------------------------------------------

\begin{document}

\maketitle

\tableofcontents

\section{基本}
  \subsection{図}
    \subsubsection{図1つ}
    図の挿入
    \begin{verbatim}
      \fig{images/hoge.png}{test}{a}{H}{width=1cm}
    \end{verbatim}
    \fig{images/hoge.png}{test}{a}{H}{width=1cm}

    図の参照
    \begin{verbatim}
      \cref{fig:a}
    \end{verbatim}
    \cref{fig:a}


  \subsection{表}
    表の挿入
    \begin{verbatim}
      \begin{tab}{H}{hoge}{|l|l|l|}{a} \hline
        衝突パラメータ & シミュレーションで得た屈折角 & 屈折角の理論値 \\ \hline
        $0.9  \cdot  0.00464918$ & 0.000009384343336 & 0.000009437262261 \\
        $1.0  \cdot  0.00464918$ & 0.000008441030730 & 0.000008493536035 \\
        $1.25 \cdot  0.00464918$ & 0.000006743259391 & 0.000006794828828 \\
        $1.5  \cdot  0.00464918$ & 0.000005611538341 & 0.000005662357357 \\
        $2.0  \cdot  0.00464918$ & 0.000004197223722 & 0.000004246768018 \\ \hline
      \end{tab}
    \end{verbatim}
    \begin{tab}{H}{hoge}{|l|l|l|}{a} \hline
      衝突パラメータ & シミュレーションで得た屈折角 & 屈折角の理論値 \\ \hline
      $0.9  \cdot  0.00464918$ & 0.000009384343336 & 0.000009437262261 \\
      $1.0  \cdot  0.00464918$ & 0.000008441030730 & 0.000008493536035 \\
      $1.25 \cdot  0.00464918$ & 0.000006743259391 & 0.000006794828828 \\
      $1.5  \cdot  0.00464918$ & 0.000005611538341 & 0.000005662357357 \\
      $2.0  \cdot  0.00464918$ & 0.000004197223722 & 0.000004246768018 \\ \hline
    \end{tab}

    表の参照
    \begin{verbatim}
      \cref{tab:a}
    \end{verbatim}
    \cref{tab:a}


  \subsection{プログラム}
    \begin{verbatim}
    \end{verbatim}


  \subsection{その他}
    改ページ
    \begin{verbatim}
      \clearpage
    \end{verbatim}
    \clearpage

    リンクされる文字
    \begin{verbatim}
      \hypertarget{name}{text}
    \end{verbatim}
    \hypertarget{name}{text}

    リンク
    \begin{verbatim}
      \hyperlink{name}{text}
    \end{verbatim}
    \hyperlink{name}{text}


\section{数式関連}
  \subsection{定義}
    \begin{verbatim}
      \begin{dfn}
        \label{dfn:hoge}
        test
      \end{dfn}
    \end{verbatim}
    \begin{screen}
      \begin{dfn}
        \label{dfn:hoge}
        test
      \end{dfn}
    \end{screen}

    参照
    \begin{verbatim}
      \cref{dfn:hoge}
    \end{verbatim}
    \cref{dfn:hoge}

  \subsection{定理}
    \begin{verbatim}
      \begin{thm}
        \label{thm:hoge}
        test
      \end{thm}
    \end{verbatim}
    \begin{thm}
      \label{thm:hoge}
      test
    \end{thm}

    参照
    \begin{verbatim}
      \cref{thm:hoge}
    \end{verbatim}
    \cref{thm:hoge}

  \subsection{証明}
    \begin{verbatim}
      \begin{proof}
        \label{proof:hoge}
        test
      \end{proof}
    \end{verbatim}
    \begin{proof}
      \label{proof:hoge}
      test
    \end{proof}

    参照
    \begin{verbatim}
      \cref{proof:hoge}
    \end{verbatim}
    \cref{proof:hoge}

  \subsection{数式}
    \subsubsection{数式環境}
      1行だけの数式
      \begin{verbatim}
        \begin{equation}
          \label{eq:test}
          y = x^2
        \end{equation}
      \end{verbatim}
      \begin{equation}
        y = x^2 \label{eq:test1}
      \end{equation}

      複数行の数式(\&で位置揃え可能だが偶数個目の\&でスペースが入る)
      \begin{equation}\begin{aligned}
        y      &= x^2 & z     &= 10 \\
        \alpha &= y   & \beta &=20
      \end{aligned}\end{equation}
      \begin{equation}
        \left.
        \begin{aligned}
          u_t + (u \cdot \nabla)u + \nabla p &= f \\
          \operatorname{div} u               &= 0
        \end{aligned}
        \right\} 
        \quad \text{Euler equation}
      \end{equation}

      複数行の数式(\&で位置揃え可能、スペースの入らないalign、行ごとの\&の数+1/2を計算していれる必要がある)
      \begin{equation}\begin{alignedat}{2}
        y      &= x^2 & z     &= 10 \\
        \alpha &= y   & \beta &=20
      \end{alignedat}\end{equation}
      \begin{equation}
        \left\{
        \begin{alignedat}{4}
            x &+2y & &+3z & &= 1   &        & \text{(A)} \\
          -3x &+ y & &- z & &= 2   &        & \text{(B)} \\
           2x &    & &- z & &= -1  & \qquad & \text{(C)}
        \end{alignedat}
        \right.
      \end{equation}

      複数行の環境の特殊なもの(各行で\&が1度しか使えず、強制的にスペースが入り左に波括弧が入る)
      \begin{equation}
        f(x) =
        \begin{dcases}
          e^{-1/x}, & x > 0 \\
          0,        & x \le 0
        \end{dcases}
      \end{equation}
      \&以降が数式環境にならないdcases
      \begin{equation}
        f(x) =
        \begin{dcases*}
          e^{-1/x}, & if $x>0$ \\
          0,        & otherwise
        \end{dcases*}
      \end{equation}

      式の列挙
      \begin{equation}\begin{gathered}
        (\sin x)' = \cos x, \quad (\cos x)' = -\sin x, \quad (\tan x)' = \frac{1}{\cos^2 x} \\
        (e^x)' = e^x, \quad (\log x)' = \frac{1}{x}
      \end{gathered}\end{equation}
      \begin{equation}
        S := \left\{ x \in \mathbb{R}^3 ;
        \begin{gathered}
          x_1 \ge 0, 0 \le x_2 \le 1, \\
          x_1 + x_2 + x_3 = 1. 
        \end{gathered} \right\}
      \end{equation}
      \begin{gather}
        \sin(2x) = 2 \sin x \cos x, \\ 
        \begin{split}
            \cos(2x) &= \cos^2 x - \sin^2 x \\
                     &= 1 - 2 \sin^2 x \\
                     &= 2 \cos^2 x - 1,
        \end{split} \\
        \tan(2x) = \frac{2 \tan x}{1 - \tan^2 x}
      \end{gather}

    \subsubsection{集合}
      括弧や縦棒の大きさが自動で代わるマクロ
      \begin{equation}
        \Set{ x \in P | x \geq p }
      \end{equation}
      \begin{equation}
        \Set{ x\in \R^n | \sum_{i=1}^n x_i^2 = 1 }
      \end{equation}

    \subsubsection{微分}
      普通の常微分
      \begin{equation}
        \diff{f(x)}{x} \label{eq:test2}
      \end{equation}
      \begin{equation}
        \diff*{\diff{f(x)}{x}}{x} \label{eq:test2'}
      \end{equation}

      偏微分
      \begin{equation}
        \diffp{f(x, y)}{x} \label{eq:test3}
      \end{equation}
      \begin{equation}
        \diffp{f(x, y)}{y, x} \label{eq:test4}
      \end{equation}

      高階微分
      \begin{equation}
        \diff[n]{f(x}{x} \label{eq:test5}
      \end{equation}
      \begin{equation}
        \diffp[a, b]{f(x, y)}{y, x} \label{eq:test5'}
      \end{equation}
      \begin{equation}
        \diffp[a, b][N]{f(x, y)}{y, x} \label{eq:test6}
      \end{equation}

      導関数値
      \begin{equation}
        \diff{f(x)}{x}[x=a] \label{eq:test7}
      \end{equation}
      \begin{equation}
        \diffp{f(x, y)}{x, y}[x=a, y=b] \label{eq:test8}
      \end{equation}

      偏微分の周りに括弧をつける
      \begin{equation}
        \diffp{L}{q_i}-\diff*{\diffp{L}{\dot{q}_i}[]}{t}=0 \label{eq:test9}
      \end{equation}



    数式の参照
    \begin{verbatim}
      \cref{eq:test1}
    \end{verbatim}
    \cref{eq:test1}

    \begin{verbatim}
      \cref{eq:test9}
    \end{verbatim}
    \cref{eq:test9}


\end{document}

